\chapter{Discusión}
\label{chap:results}

Tal y como se estableció en los objetivos, se utilizó un filtro media móvil para atenuar el ruido y resaltar las características de la señal, una vez finalizada la fase de experimentación y obtenidos los resultados se determinó que este filtro no es suficiente para extraer características relevantes en la señal, debido a que las señales obtenidas por el módulo AD8232, por ende se plantea aplicar transformada de Fourier o de Wavelet para analizar las señales del conjunto de datos tanto de entrenamiento como de validación, dado que estas dos técnicas de filtrado ofrecen ventajas como la preservación de información de alta frecuencia, adaptabilidad a diferentes escalas, eficiencia computacional y supresión selectiva de ruido en comparación con el filtro aplicado.

En cuanto a la evaluación del modelo de clasificación para señales ECG normales y anormales ha mostrado un porcentaje bajo en cuanto a exactitud, es decir, la proporción de todas las predicciones correctas realizadas, ya que solo alcanzó un 64.39\% al realizar esta tarea, pero se obtuvo un mayor porcentaje ("69.3069") al evaluar la sensibilidad, métrica que mide la capacidad del modelo para identificar correctamente los casos positivos que clasificó la red neuronal recurrente.

Como trabajo a futuro se plantea aplicar ya sea transformada de Fourier o de Wavelet para analizar los datos adquiridos por el módulo AD8232 y las señales del conjunto de datos tanto de entrenamiento como de validación, ya que de acuerdo con algunos autores al convertir una señal en función del tiempo en una representación en función de la frecuencia permitirá identificar patrones característicos. Por lo tanto, utilizar este tipo de transformadas tiene potencial para obtener información relevante para futuros proyectos.
