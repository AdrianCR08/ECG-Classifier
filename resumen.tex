\chapter*{Resumen}
% capítulo sin numerar

\selectlanguage{spanish}
El proyecto de adquisición y clasificación de señales electrocardiográficas utilizando un sensor AD8232 y una red neuronal recurrente es una idea original que se enfoca en la obtención de señales electrocardiográficas mediante sensores para su clasificación en Normal y Anormal.
Para garantizar la validez y confiabilidad del proyecto, se han establecido métricas de evaluación como la \textbf{exactitud}, \textbf{sensibilidad}, \textbf{especificidad} y \textbf{clasificación errónea} de la red neuronal. 
Estas métricas son esenciales para evaluar la eficacia del modelo de predicción y asegurar que las decisiones cuentan con precisión y confiabilidad.
La aportación del proyecto a la disciplina de Ingeniería en Computación se centra en el sistema embebido de obtención de señales y su tratamiento, lo que permite una integración natural con la red neuronal.\\
Además, el uso de herramientas especializadas como MATLAB para el desarrollo de la red neuronal representa una innovación en la aplicación de la inteligencia artificial y el machine learning en la detección temprana de enfermedades cardíacas.\\
El proyecto se enfoca en áreas como sistemas embebidos e inteligencia artificial, lo que representa una oportunidad para explorar nuevas aplicaciones de la tecnología en el campo de la ingeniería. 
En general, el proyecto aspira poder mejorar la obtención de señales para su posterior clasificación, lo que podría tener un impacto significativo en la contribución al conocimiento.

% Colocar cinco palabras clave
{\keywordssp{Señales electrocardiográficas, Aprendizaje Automático, Sistema Embebido, Clasificación, Inteligencia Artificial.}}

\chapter*{Abstract}
% capítulo sin numerar

\selectlanguage{english}

The project for the acquisition and classification of electrocardiographic signals using an AD8232 sensor and a recurrent neural network is an original idea that focuses on obtaining electrocardiographic signals through sensors for their classification into Normal and Abnormal. To guarantee the validity and reliability of the project, evaluation metrics have been established such as accuracy, sensitivity, specificity and neural network misclassification.\\ These metrics are essential for evaluating the effectiveness of the forecasting model and ensuring that decisions are accurate and reliable. The contribution of the project to the discipline of Computer Engineering focuses on the embedded system for obtaining signals and their treatment, which allows a natural integration with the neural network. In addition, the use of specialized tools such as MATLAB for the development of the neural network represents an innovation in the application of artificial intelligence and machine learning in the early detection of cardiac diseases.
The project focuses on areas such as embedded systems and artificial intelligence, which represents an opportunity to explore new applications of the technology in the field of engineering. In general, the project aspires to be able to improve the obtaining of signals for their subsequent classification, which could have a significant impact on the contribution to knowledge.


% keywords in English
{\keywords{Electrocardiographic signals, Machine Learning, Embedded System, Predictions, Artificial Inteligence}}

\selectlanguage{spanish}