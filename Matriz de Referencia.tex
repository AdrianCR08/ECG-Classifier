% Please add the following required packages to your document preamble:
% \usepackage{lscape}
% \usepackage{longtable}
% Note: It may be necessary to compile the document several times to get a multi-page table to line up properly
\begin{landscape}
    \footnotesize
    \begin{longtable}{llll}
    \caption{Matriz de Referencias }
    \label{tab:Mat_Ref}\\
    \hline
    \multicolumn{1}{c}{\textbf{Fuente}} & \multicolumn{1}{c}{\textbf{Resumen}} & \multicolumn{1}{c}{\textbf{Método}} & \multicolumn{1}{c}{\textbf{Áreas de Oportunidad}} \\ \hline
    \endfirsthead
    %
    \endhead
    %
    \begin{tabular}[c]{@{}l@{}}Aplicación Android para \\ la adquisición inalámbrica y \\ visualización \\ de señales biomédicas.\\ Smith, R.E., \\ Socarrás, B. N., \& \\ Vázquez, C. R.(2022).\end{tabular} & \begin{tabular}[c]{@{}l@{}}En la actualidad se han \\ implementado diversas \\ aplicaciones (software) \\ móviles con  afines a la\\  monitorización\\ inalámbrica de señales para \\ adquirir, visualizar y \\ almacenar variables biomédicas.\end{tabular} & \begin{tabular}[c]{@{}l@{}}Para la construcción de dicha\\ aplicación se emplea el lenguaje\\  de programación Java,\\ por lo que se basa de recursos\\  como botones, listas, imágenes, etc. \\ Y para montarla en el sistema \\ operativo Android se emplea \\ Android Studio \\ versión 4.3.1 del IDE Intellij.\end{tabular} & \begin{tabular}[c]{@{}l@{}}Incluir más funcionalidades \\ relacionadas al procesamiento\\ y análisis de los bioseñales:\\ Implementación de algún \\ algoritmo de inteligencia \\ artificial para realizar \\ clasificaciones o predicciones \\ de enfermedades\end{tabular} \\ \hline
    \begin{tabular}[c]{@{}l@{}}Diseño de un \\ Microsistema para\\  Adquisición de \\ Señales Cardíacas \\ Usando FPAAs.\\ Rodriguez, P., \\ Castro, H.,  \\ Pinedo, C. \& \\ Velasco, J. (2023).\end{tabular} & \begin{tabular}[c]{@{}l@{}}En el corazón se producen \\ variaciones de voltaje, \\ gracias a la implementación\\  de microelectrodos se pueden \\ adquirir dichos datos como\\ señales bioeléctricas y \\ posteriormente representarlos\\ gráficamente como señales ECG, \\ las cuales presentan distintas\\  regiones (P, Q, R, S y T).\end{tabular} & \begin{tabular}[c]{@{}l@{}}CARDIOCEL se concibe como \\ un microsistema electrónico \\ económico que permite \\ adquirir señales ECG.\\ Consta de dos partes importantes:\\ - Hardware (adquirir, procesar \\ señales analógicas, digitalizarlas\\  y transmitirlas).\\ - Software (procesamiento y \\ visualización de las ECG).\end{tabular} & \begin{tabular}[c]{@{}l@{}}Rediseño del sistema para visualizar\\ simultáneamente dos o más \\ derivaciones de ECG.\\ Implementar otro tipo\\  de comunicación inalámbrica\\ para que el microsistema \\ CARDIOCEL se pueda integrar \\ a una red de telemetría hospitalaria.\end{tabular} \\ \hline
    \begin{tabular}[c]{@{}l@{}}Procedimiento\\ para la detección de \\ señales balistocardiográficas\\ y sistema que lo implementa\\ OMPI. Palma, A. et al. (2020)\end{tabular} & \begin{tabular}[c]{@{}l@{}}La implementación de\\ balistocardiograma (BCG)\\ tiene la ventaja de no\\ usar electrodos u \\ otros elementos que se\\ dieran al cuerpo para \\ monitorear los latidos \\ del corazón, pero si \\ emplean sensores \\ capacitivos con la finalidad \\ de ofrecer más flexibilidad\\  y comodidad.\end{tabular} & \begin{tabular}[c]{@{}l@{}}Para adquirir dicha información \\ existen diversos métodos:\\ - Filtro paso baja o FPB.\\ - Transformada de Wavelet.\\ - Técnicas de aprendizaje \\ no supervisado.\end{tabular} & \begin{tabular}[c]{@{}l@{}}Implementación de otras técnicas\\ (diferentes a FPB.) para analizar \\ señales cardíacas.\end{tabular} \\ \hline
    \begin{tabular}[c]{@{}l@{}}A comprehensive study \\ of machine learning \\ for predicting \\ cardiovascular disease\\  using Weka and SPSS tools\\ Abuhaija, B. et. al. (2023)\end{tabular} & \begin{tabular}[c]{@{}l@{}}La clasificación es considerada\\ como un sistema de \\ aprendizaje supervisado\\ que usa datasets \\ etiquetados representando\\ predicciones, \\ ANN y MLP son una de \\ las mejores técnicas para \\ clasificar datos organizados\\  en forma tabular.\end{tabular} & \begin{tabular}[c]{@{}l@{}}Los métodos de machine \\ learning más comunes \\ para predecir la presencia\\  de ataques al corazón \\ incluyen k-Nearest Neighbor\\ con poca precisión y Random Forest \\ con la mayor precisión con un 90\%. \\ Otras como CNN \\ (Convolutional Neural Network) \\ con un 82\% de precisión.\end{tabular} & \begin{tabular}[c]{@{}l@{}}Analizar con técnicas exploratorias\\ las señales ECG para identificar \\ enfermedades cardíacas por \\ medio de clusters o redes \\ neuronales artificiales.\end{tabular} \\ \hline
    \begin{tabular}[c]{@{}l@{}}Intelligent Sustainable Systems: \\ Proceedings of ICISS 2021 \\ (Lecture Notes in \\ Networks and Systems, 213)\\ Jennifer S. Raj. et. al. (2021)\end{tabular} & \begin{tabular}[c]{@{}l@{}}Una Red Neuronal Profunda (DNN)\\ permite robustez para variaciones \\ de datos, así como la generalización \\ para múltiples aplicaciones además\\ de ser escalable para más datos.\end{tabular} & \begin{tabular}[c]{@{}l@{}}Los dataset más completos\\ para la predicción de \\ enfermedades cardiovasculares\\ son Statlog y Cleveland.\\ Para su preprocesado \\ cada uno fue dividido \\ en un radio de 70:30, \\ es decir, el 70\% está \\ destinado a entrenamiento\\  y el 30\% restante para \\ probar el modelo.\end{tabular} & \begin{tabular}[c]{@{}l@{}}Emplear de señales \\ electrocardiográficas \\ para la clasificación \\ y predicción de \\ enfermedades \\ cardiovasculares\end{tabular} \\ \hline
    \begin{tabular}[c]{@{}l@{}}Análisis Espacial\\ De Las Enfermedades \\ Cardiovasculares \\ En México: \\ Modelos Predictivo\\ Gonzélez Medina, L. E., (2022).\end{tabular} & \begin{tabular}[c]{@{}l@{}}Dentro del modelo \\ predictivo se considera\\ el manejo y desarrollo \\ identificado como \\ campo aleatorio que\\  es un proceso estocástico \\ que toma valores en un\\ espacio euclidiano, ya \\ que dentro de su proceso \\ está definido sobre un \\ espacio parametral de \\ dimensión uno.\end{tabular} & \begin{tabular}[c]{@{}l@{}}La estadística de los datos \\ permite la manipulación \\ y clasificación, las diversas \\ predicciones se manejan \\ por medio de áreas las \\ cuales por ejemplo \\ permiten la identificación \\ del riesgo relativo de \\ padecer una enfermedad \\ cardiovascular, número \\ de casos, casos esperados.\end{tabular} & \begin{tabular}[c]{@{}l@{}}Identificar y conocer \\ el manejo espacial de \\ datos dentro de un \\ sistema de modelos \\ predictivos de \\ enfermedades \\ cardiovasculares.\end{tabular} \\ \hline
    \begin{tabular}[c]{@{}l@{}}Adquisición de \\ Señales analógicas de \\ instrumentación con \\ logo! soft V8.3 \\ Mediante Generador \\ de Señales y el sensor PT100,\\ Ortega Ordoñez, R.C. et al. (2023)\end{tabular} & \begin{tabular}[c]{@{}l@{}}Por medio del análisis de \\ señales permiten la \\ visualización de los \\ diversos datos de un \\ manejo de datos altos \\ y bajos para la adquisición\\  de la predicción, ya que \\ este mecanismo permite \\ la identificación de \\ enfermedades cardiovasculares \\ obteniendo así las variables \\ identificadas para tener \\ nuevos resultados.\end{tabular} & \begin{tabular}[c]{@{}l@{}}La adquisición de datos \\ se obtiene por medio \\ de las señales o de igual\\ manera se puede \\ utilizar el software \\ LabVIEW que permite \\ manejar un entorno \\ virtual para el manejo\\ de análisis y predicción\\ de personas con enfermedades \\ cardiovasculares.\end{tabular} & \begin{tabular}[c]{@{}l@{}}Aplicar filtros pasa bajas,\\ altas, o pasa bandas para\\ determinar el filtro más \\ adecuado en este ámbito.\end{tabular} \\ \hline
    \begin{tabular}[c]{@{}l@{}}Construcción de un \\ sistema electrocardiográfico \\ con conexión inalámbrica\\  a teléfonos inteligentes\\ Andrés , R.M. \\ Willian Andrés,C.C.\\  et al. (2021)\end{tabular} & \begin{tabular}[c]{@{}l@{}}Para el diagnóstico y tratamiento\\ primeramente es indispensable la \\ realización de un examen médico \\ electrocardiograma, el cual dentro\\ de su proceso registra la actividad \\ eléctrica del corazón, dicho proceso\\ permite el análisis para la predicción \\ en la persona si tiene alguna \\ enfermedad  cardiovascular.\end{tabular} & \begin{tabular}[c]{@{}l@{}}Red de sensores inalámbricos \\ ECG o WSN son un conjunto\\ de dispositivos de tamaño pequeño \\ interconectados entre sí, su \\ principal funcionamiento es que\\ monitorea variables físicas para\\ enviar información del entono\\ de manera inalámbrica hasta el \\ dispositivo en donde llegara la\\ señal o servidor.\end{tabular} & \begin{tabular}[c]{@{}l@{}}Aplicar análisis de \\ de señales ECG \\ dentro del sistema \\ electrocardiográfico\end{tabular} \\ \hline
    \begin{tabular}[c]{@{}l@{}}Design and \\ implementation of \\ low cost ECG \\ monitoring system\\ for the patient \\ using smartphone.\\ Ahamed, Md \\ Asif, Hasan, \\ Md. Kamrul, \\ \& Alam, Md. (2015).\end{tabular} & \begin{tabular}[c]{@{}l@{}}El dispositivo se \\ conecta a los electrodos \\ de ECG colocados en \\ el cuerpo humano y \\ envía los datos de ECG \\ adquiridos a través de \\ Bluetooth al smartphone.\end{tabular} & \begin{tabular}[c]{@{}l@{}}La aplicación utiliza \\ algoritmos de \\ procesamiento de \\ señales para filtrar \\ la señal de ECG y\\ visualizarla en tiempo real.\end{tabular} & \begin{tabular}[c]{@{}l@{}}Realizar evaluaciones \\ clínicas para determinar \\ la precisión y la eficacia \\ del sistema de monitoreo \\ de ECG en comparación \\ con sistemas convencionales.\end{tabular} \\ \hline
    \begin{tabular}[c]{@{}l@{}}Classification of \\ ECG Arrhythmia \\ Using Deep Learning.\\ Mohammadpour, A., \\ \& Sadabadi, F. (2019).\end{tabular} & \begin{tabular}[c]{@{}l@{}}Se describe un enfoque basado en \\ una red neuronal convolucional\\ profunda (CNN) para la \\ clasificación de arritmias cardíacas.\\ La base de datos utilizada\\  en el estudio es la base de\\ datos de arritmia del MIT-BIH, \\ que contiene 48 registros de \\ señales de ECG de pacientes\\  con diferentes tipos de arritmias. \\ El conjunto de datos se divide \\ en conjuntos de entrenamiento \\ y prueba en una proporción de 70:30.\end{tabular} & \begin{tabular}[c]{@{}l@{}}Los resultados del estudio \\ muestran que la CNN \\ propuesta obtuvo una \\ tasa de precisión del 98,4\% \\ en la clasificación de las \\ arritmias, lo que indica \\ que el enfoque propuesto \\ es altamente efectivo en \\ la detección y clasificación \\ de arritmias cardíacas.\end{tabular} & \begin{tabular}[c]{@{}l@{}}Enfocar el estudio\\ en la clasificación de \\ cuatro tipos de arritmias, \\ pero existen muchos \\ otros tipos de arritmias\\ que podrían ser \\ abordados en futuros estudios.\end{tabular} \\ \hline
    \begin{tabular}[c]{@{}l@{}}Real-Time ECG \\ Monitoring System \\ Using Raspberry Pi \\ and Arduino for \\ Healthcare Applications.\\ Zolkipli, M. Z.,\\ Yusof, N. M., \\ \& Ali, N. M. (2018).\end{tabular} & \begin{tabular}[c]{@{}l@{}}El microcontrolador \\ Arduino es el encargado \\ de procesar la señal de \\ ECG y de enviar los \\ datos a través del \\ puerto serial al Raspberry Pi.\\ En el Raspberry Pi, \\ los datos de ECG \\ son procesados y analizados \\ por medio de un programa en\\  Python.\end{tabular} & \begin{tabular}[c]{@{}l@{}}El sistema desarrollado \\ fue evaluado utilizando \\ señales de ECG de\\ voluntarios y se \\ encontró que el \\ sistema es capaz \\ de adquirir y procesar\\ las señales de ECG \\ en tiempo real con \\ una alta precisión.\end{tabular} & \begin{tabular}[c]{@{}l@{}}Investigar cómo mejorar \\ el rendimiento del sistema\\ en ambientes ruidosos, \\ como en presencia de otros \\ dispositivos médicos o \\ en entornos hospitalarios.\end{tabular} \\ \hline
    \begin{tabular}[c]{@{}l@{}}Aplicaciones de la \\ inteligencia artificial \\ en cardiología: \\ el futuro ya está aquí.\\ Dorado-Díaz, P.I., \\ Sampedro-Gómez, J., \\ Vicente-Palacios, V., \\ Sánchez, P.L., 2019.\end{tabular} & \begin{tabular}[c]{@{}l@{}}Existen dos tipos de técnicas de \\ aprendizaje AA: supervisado y \\ no supervisado. En el aprendizaje \\ supervisado, se cuenta con un \\ conjunto de datos etiquetados \\ para predecir una variable de \\ respuesta específica, mientras \\ que en el no supervisado, no \\ se tiene información sobre la \\ variable de respuesta. Los \\ algoritmos de clasificación y \\ regresión se utilizan para la \\ predicción en el aprendizaje \\ supervisado.\end{tabular} & \begin{tabular}[c]{@{}l@{}}El proceso de construcción de un\\ modelo de Aprendizaje Automático \\ (AA) no se limita a aplicar un \\ algoritmo de aprendizaje a una \\ base de datos, sino que es un \\ proceso más complejo que \\ involucra varios pasos.\\ En primer lugar, se parte de los datos\\ brutos y se realiza un preprocesado para\\ convertirlos en datos estructurados.\end{tabular} & \begin{tabular}[c]{@{}l@{}}Identificar patrones que los seres \\ humanos no serían capaces de \\ detectar. Sin embargo existe una fuerte \\ dependencia de la calidad y cantidad \\ de los datos utilizados para entrenar el\\ modelo.\end{tabular}
    \end{longtable}
    \end{landscape}