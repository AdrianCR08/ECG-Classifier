\chapter{Resultados}
\label{chap:results}
Como ya se mencionó la \textbf{Exactitud} alcanzada por el modelo en la fase de entrenamiento es del $64.39\%$, en esta sección se abordara a detalle el resultado obtenido por el modelo.

La figura \ref{fig:Mat_Conf} muestra la matriz de confusión del modelo de clasificación.

\begin{figure}[!ht]
    \centering
    \includegraphics[width = 0.7\textwidth]{images/img34.png}
    \caption{Matriz de confusión del modelo (Entrenamiento)}
    \label{fig:Mat_Conf}
\end{figure}

La Exactitud del modelo esta dada por la formula \ref{eq:accuracy}

\begin{equation}
    E = \frac{2065 + 3651}{2065 + 3651 + 787 + 2373} \times 100 = 64.3983
\end{equation}

Es decir, el modelo fue capaz de clasificar un 64.39\% de las señales ECG en la categoría correcta en el conjunto de datos de entrenamiento.

La Sensibilidad del modelo esta dada por \ref{eq:Sensitivity}

\begin{equation}
    S = \frac{2065}{2065 + 787} \times 100 = 34.2795
\end{equation}

Es decir, el modelo ha sido capaz de clasificar $34.27\%$ de las muestras positivas en relación con el total de muestras positivas del conjunto.

La Especificidad del modelo esta dada por \ref{eq:specifity}

\begin{equation}
    Es = \frac{3651}{3651 + 2373} \times 100 = 60.6075
\end{equation}

En otras palabras, el modelo clasificó un $60.6075\%$ de muestras negativas en relación con el total.


En cuanto a los resultados de la \textbf{Validación} alcanzada por el modelo es del $61.93\%$, en esta sección se abordara a detalle el resultado obtenido por el modelo.

La figura \ref{fig:Mat_Conf_Val} muestra la matriz de confusión del modelo de clasificación.

\begin{figure}[!ht]
    \centering
    \includegraphics[width = 0.7\textwidth]{images/img35.png}
    \caption[]{Matriz de confusión del modelo (Validación)}
    \label{fig:Mat_Conf_Val}
\end{figure}


\begin{itemize}
    \item Exactitud
\end{itemize}

\begin{equation}
    E = \frac{210 + 397}{210 + 397 + 93 + 280} \times 100 = 61.9387
\end{equation}

Es decir, el modelo fue capaz de clasificar un $61.93\%$ de las señales ECG en la categoría correcta en el conjunto de datos de entrenamiento.

\begin{itemize}
    \item Sensibilidad
\end{itemize}

\begin{equation}
    S = \frac{210}{210 + 93} \times 100 = 69.3096
\end{equation}

Es decir, el modelo ha sido capaz de clasificar $69.30\%$ de las muestras positivas en relación con el total de muestras positivas del conjunto en la fase de validación.

\begin{itemize}
    \item Especificidad
\end{itemize}

\begin{equation}
    Es = \frac{397}{397 + 280} \times 100 = 58.6410
\end{equation}

En otras palabras, el modelo clasificó un $58.64\%$ de muestras negativas en relación con el total en la fase de validación.


