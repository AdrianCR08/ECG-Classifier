\chapter{Instructivo}
\label{chap:instrucciones}

Este capítulo no forma parte del protocolo; se proporciona con la finalidad de proporcionar indicaciones generales sobre el uso de los principales elementos de \LaTeX. 

\section{Contenido del documento}
\label{sec:contenido}

A continuación se presentan algunos lineamientos y recomendaciones a tomar en cuenta para la elaboración del documento con algunos ejemplos básicos para utilizar contenido de diferentes tipos, incluyendo tablas, figuras, citas y referencias bibliográficas, ecuaciones, listas y notas al pie de página. 

\subsection{Estilo de redacción}
\label{sec:estilo}

Se deben explicar, ilustrar, discutir, las evidencias, debidamente citadas, que respaldan la teoría que se presenta. El estilo de redacción (de carácter académico o profesional) se realiza de manera impersonal (tercera personal del singular), y preferentemente en tiempo presente. Se recomienda utilizar párrafos cortos (de 4 a 7 enunciados en promedio), y cada párrafo debe representar una idea concreta.

\subsection{Citas y referencias}
\label{sec:citas}

En el documento se deberán incluir citas a fuentes diversas, confiables, verificables y vigentes, en el formato predefinido en este documento o en el indicado por el docente. 

Las citas preferidas son los artículos de investigación en el formato de autor-año, tal como aparecen en este párrafo de ejemplo; nótese que el punto final se coloca después de las referencias y no antes \citep{bib1,bib2}. En ambos ejemplos solamente aparecen los primeros autores, ya que cuando son varios, la abreviatura \textit{et al.} significa: \textit{y colaboradores}.

Este es un ejemplo de una cita, en donde la fuente se coloca usualmente al final del enunciado \citep{bib8}. Pueden existir varias referencias que ayuden a complementar una idea, por lo que también aparecen juntas \citep{bib3,bib11,bib12}. En un párrafo puede haber una o más referencias, si es necesario.  

Cuando es importante hacer mención del autor, por ser una autoridad en la materia, o por variar en el estilo de redacción, la manera de citarlo es un poco diferente. En este ejemplo se supone que es importante lo que dijeron \citet{bib4}, ya que mencionan que algo es importante para la investigación. O quizá lo que hizo \citet{bib12} para enriquecer la discusión.  

En estos casos, se coloca el autor y entre paréntesis el año, pero \LaTeX ~ya lo hace, solo hay que usar el estilo correcto de citación, como los que se mostraron con los comandos \verb|\citep{referencia}| que proporciona una cita del tipo \textbf{(autor, año)}, y \verb|\citet{referencia}|, que aparece como  \textbf{autor (año)}.  

Existen diversos ejemplos de documentos o fuentes que se deben citar, por ejemplo:

\begin{itemize}
\item Parte de libro, indicando páginas \citep[págs. 10--13]{bib5}. 
\item Parte de un libro especificando el capítulo \citep[cap. 4]{bib_chap}
\item Capítulo de una serie de volúmenes \citep{bib7}.
\item Tesis \citep{bib_tesis} (en este caso, tome nota del tipo de entrada que se usa en el archivo biblio.bib).
\item Libro editado \citep{bib6}.
\item Actas de conferencia \citep{bib7}.
\item Ponencia o charla \citep{bib8}.
\item Artículo en proceso (pre-print) \citep{bib10}
\item Un banco de imágenes o un conjunto de archivos \citep{bib9}.
\end{itemize}

No olvidar que las citas textuales deben ir entre comillas y, cuando sean superiores a 40 palabras, se deben colocar en un párrafo independiente. Indicar siempre su referencia, especificando el número de página, o el lugar de la fuente de donde se obtuvo. Asimismo, no es conveniente utilizar demasiadas citas textuales; en cambio, procesar la información y escribirla con palabras propias, procurando incluir argumentos e ideas originales. 

% \citeA{bib1} 
% uso con apacite

% \shortcite{bib4} vs \fullcite{bib4}.
% uso con apacite

%\parencite{bib2}.
% biber
% \citeauthor[p.1]{bib1} y \citeyear{bib7}.
% biber

\subsection{Notas al pie}
\label{sec:notas}

Las notas al pie, como en este ejemplo\footnote{Detalles relevantes que pueden contener una referencia \citep{bib3} o no.}, son útiles para aportar datos importantes que no están necesariamente ligados con la redacción; aportan información complementaria y pueden incluir referencias (solo si lo ameritan, ya que puede ser información conocida o un comentario del autor). Se pueden incluir páginas web, repositorios, o bases de datos, que no necesariamente requieren ser citados.

\subsection{Listas}
\label{sec:listas}

Algunos ejemplos de listas o enumeraciones que se pueden usar como punto de partida para cierta información, se presentan en este apartado. En la siguiente lista se utilizan viñetas, con diferentes niveles, para agrupar los elementos que contiene según su jerarquía o clase. 

\begin{itemize}
\item Edad de Piedra
	\begin{itemize}
	\item Paleolítico
		\begin{itemize}
		\item Paleolítico inferior
		\item Paleolítico medio
		\item Paleolítico superior
	\end{itemize}
	\item Mesolítico
	\item Neolítico
	\end{itemize}
\item Edad del Cobre
\item Edad del Bronce
\item Edad del Hierro
\end{itemize} 

Por otra parte, en el siguiente ejemplo se presenta una lista numerada, con la misma estructura de la lista anterior. 

\begin{enumerate}
\item Edad de Piedra
	\begin{enumerate}
	\item Paleolítico
		\begin{itemize}
		\item Paleolítico inferior
		\item Paleolítico medio
		\item Paleolítico superior
		\end{itemize}
	\item Mesolítico
	\item Neolítico
	\end{enumerate}
\item Edad del Cobre
\item Edad del Bronce
\item Edad del Hierro
\end{enumerate} 

\subsection{Tablas}
\label{sec:tablas}

Ejemplo de una tabla. Siempre debe mencionarse en el texto una tabla antes de ser presentada; por ejemplo, ver tabla \ref{tab:ejemplo}. Es conveniente explicar también el contenido de la tabla, ya que no debe dejarse al lector la interpretación del mismo. Tomar nota de los modificadores de posición, así como la posibilidad de usar un generador de tablas en línea para facilitar su edición\footnote{\url{https://www.tablesgenerator.com/}}.

\begin{table}[!ht]
\centering
\caption{Descripción breve, concisa, pero eficaz del contenido de la tabla} 
% En una tabla, la descripción se coloca encima de ella
\label{tab:ejemplo}
\begin{tabular}{c c c c} 
\hline
Col1 & Col2 & Col2 & Col3 \\ [0.5ex] 
\hline\hline
1 & 6 & 87837 & 787 \\ 
2 & 7 & 78 & 5415 \\
3 & 545 & 778 & 7507 \\
4 & 545 & 18744 & 7560 \\
5 & 88 & 788 & 6344 \\ [1ex] 
\hline
\end{tabular}
\end{table}
 
\subsection{Imágenes}
\label{sec:figuras}

Enseguida se presenta el ejemplo de una figura. Siembre debe mencionarse en el texto una figura antes de que aparezca en el documento, por ejemplo, en la figura \ref{fig:figuraejm}. Describir la figura y sus detalles, sin dejar al lector la interpretación de su contenido. 

\begin{figure}[!ht]
\centering
\includegraphics[width=0.25\textwidth]{imagen-ejemplo}
\caption{Descripción breve, concisa, pero eficaz de la figura}
% En una figura, la descripción se coloca debajo de ella
\label{fig:figuraejm}
\end{figure}

\subsection{Ecuaciones}
\label{ecuaciones}

Las ecuaciones se trabajan dentro del entorno de \verb|\begin(equation) y \end{equation}|. Deben mencionarse en el texto, como en la ecuación \ref{eq:eq1}, utilizan referencias cruzadas como en tablas e imágenes, y deben numerarse. 

\begin{equation}
\|\tilde{X}(k)\|^2 \leq\frac{\sum\limits_{i=1}^{p}\left\|\tilde{Y}_i(k)\right\|^2+\sum\limits_{j=1}^{q}\left\|\tilde{Z}_j(k)\right\|^2 }{p+q}
\label{eq:eq1}
\end{equation}

\noindent
siendo:
\begin{equation}
\notag
Y_\mu =  \partial_\mu - ig \frac{\lambda^a}{2} A^a_\mu
\end{equation}

Nótese en las líneas anteriores que en ocasiones se puede omitir el número de ecuación mediante el comando \verb|\notag| (e incluso su etiqueta) cuando se está usando el entorno de ecuaciones \verb|\begin(equation)| y \verb|\end{equation}| para definir un término particular, o una ecuación secundaria que deriva de una ecuación principal. Otro ejemplo similar se aprecia en la ecuación \ref{eq:eq3}. Obsérvese que en este caso se están usando los delimitadores  \verb|$ ... $| para escribir ecuaciones en linea con el texto y sin numeración, por ejemplo, para definir sus literales. 

\begin{equation}
Y_\infty = \left( \frac{m}{\textrm{GeV}} \right)^{-3}
    \left[ 1 + \frac{3 \ln(m/\textrm{GeV})}{15}
    + \frac{\ln(c_2/5)}{15} \right]
\label{eq:eq3}
\end{equation}

\noindent
donde:\\
$m$ es la masa, \\
$GeV$ es la aceleración en unidades de \textit{giga eV}, \\
$c_2$ es etcétera. 

\section{Consideraciones generales}
\label{sec:consideraciones}

Tomar en cuenta las siguientes consideraciones en la creación del documento: 

\begin{itemize}
\item Todas las abreviaturas y siglas deben ser especificadas la primera vez que aparecen en el texto.
\item No incluir abreviaturas o siglas en el título del trabajo o en los títulos de los capítulos.
\item Utilizar máximo hasta un tercer nivel en los títulos; es decir, se pueden usar \verb|\section{•}|, \verb|\subsection{•}|  y \verb|\subsubsection{•}|.
\item Todas las tablas y figuras que aparezcan en el documento deben explicarse o describirse previamente usando referencias cruzadas (\verb|\ref{fig:nombrefigura}| o \verb|\ref{tab:nombretabla}|). 
\item Todas las ecuaciones deben numerarse (salvo los casos mencionados anteriormente) y se debe usar la referencia cruzada (\verb|\ref{eq:nombreecuacion}|) para identificarlas adecuadamente. 
\item Citar la información: el no seguir debidamente las recomendaciones para las citas y el formato de citación que se ha indicado puede ser considerado un indicio de plagio, con la consecuente anulación del trabajo. 
\end{itemize}

